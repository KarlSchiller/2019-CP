\section*{Aufgabe 1}
\subsection*{Aufgabenteil a)}
Als Ansatz wurde Lagrange gewählt mit der folgenden Lagrange-Funktion
\begin{equation*}
  L = \sum_{i=1}^N \left(\frac{1}{2} m_i \dot{u_i}^2 - \frac{1}{2} k_i (u_{i+1}-u_i)^2 \right) \, ;
\end{equation*}
dabei sind $m_i$ die Massen, $k_i$ die Federkonstanten und $u_i$ die Positionen der Massen
in Normalkoordinaten.
Mit der Euler-Langrange-Gleichung
\begin{equation*}
  0 = \frac{\symup d}{\symup dt} \frac{\partial L}{\partial \dot{u_i}} - \frac{\partial L}{\partial u_i}
\end{equation*}
folgt
\begin{align}
  \frac{\symup d}{\symup dt} m_i \dot{u_i} &= -k_{i-1}(u_i - u_{i-1}) + k_i(u_{i+1}-u_i) \\
  \Leftrightarrow \ m_i \ddot{u}_i &= k_{i-1} u_{i-1} - k_{i-1} u_i + k_i u_{i+1} - k_i u_i
  \label{eqn:1} \, .
\end{align}
Nun werden die Massenmatrix $\underline{\underline{M}}$ mit
\begin{equation*}
  \underline{\underline{M}} = \begin{bmatrix}
    m_1 & 0 & \dots & \\
    0 & m_2 & 0 & \dots \\
    \vdots & 0 & \ddots & 0 \\
    0 & 0 & 0  & m_N \\
\end{bmatrix}
\end{equation*}
und die Matrix der Federkonstanten $\underline{\underline{K}}$ mit
\begin{equation*}
  \underline{\underline{K}} = \begin{bmatrix}
  -k_1 & k_1 & 0 & \dots \\
  k_1 & -k_1-k_2 & k_2 & 0 & \dots \\
  0 & k_2 & -k_2-k_3 & k_3 & 0 \\
  \vdots & 0 & \ddots & \ddots & \vdots \\
  & & & k_{N-1} & -k_{N-1} \\
  \end{bmatrix}
\end{equation*}
aufgestellt und \eqref{eqn:1} zu
\begin{align}
  \underline{\underline{M}} \, \ddot{\vec{u}} - \underline{\underline{K}} \, \vec{u} = \vec{0} \\
  \left(-\omega^2\underline{\underline{M}} - \underline{\underline{K}} \right) \vec{u} = \vec{0}
  \label{eqn:2}
\end{align}
umgestellt. Dabei wurde der Ansatz $u_i \propto \symup e^{i \omega t}$ verwendet.
Nun wird \eqref{eqn:2} in eine Eigenwertgleichung überführt:
\begin{align*}
  \left(-\omega^2 \mathbb{1} +
  \underline{\underline{M}} - \underline{\underline{K}} \right)
  \vec{u} &= \vec{0} \\
  \underline{\underline{M}} - \underline{\underline{K}} &\equiv \underline{\underline{A}} \\
  \left(\underline{\underline{A}} -\omega^2 \mathbb{1} \right) \vec{u} &= \vec{0} \, .
\end{align*}
Dann wird durch die \textsc{Eigen}-Bibliothek die Eigenwerte bestimmt, aus denen wegen
$\lambda = \omega^2$ noch die Wurzel gezogen werden muss, um die Eigenfrequenzen
des Systems zu erhalten.

\subsection*{Aufgabenteil b)}
Es wurden die angegebenen Größen implementiert und der Algorithmus durchlaufen.
Allerdings liegt die maximale Eigenfrequenz über dem angegebenen Wert.
\begin{table}
  \caption{Eigenfrequenzen}
  \label{tab:1}
  \begin{tabular}{c c c c c c c c c c}
    \toprule
    $\omega_1$ & $\omega_2$ & $\omega_3$ & $\omega_4$ & $\omega_5$ & $\omega_6$ &
    $\omega_7$ & $\omega_8$ & $\omega_9$ & $\omega_{10}$ \\
    \midrule
    1.57777 & 2.57907 & 2.99416 & 3.22374 & 3.41898 & 3.69576 & 4.04046 & 4.44884 &
    4.93309 & 5.54081 \\
    \bottomrule
  \end{tabular}
\end{table}
