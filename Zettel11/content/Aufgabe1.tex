\section*{Aufgabe 1}

\subsection*{1.a) und b)}
Das Travelling Salesman Problem wurde unter Verwendung einer Variante des Metropolis-
Algorithmus gelöst.

Hierzu sind die Ortsvektoren in einer Matrix \texttt{r} gespeichert, wobei jede
Spalte einen Ortsvektor enthält.
Die Reihenfolge der Ortsvektoren ist in einem Vektor \texttt{perm} gespeichert,
wobei \texttt{perm(i)} den Platz von Ortsvektor \texttt{i} kennzeichnet.

Zunächst wird das Programm wie gefordert mit der Funktion \texttt{sa\_init} initialisiert.
Der Abstand $\delta$ der Ortsvektoren ist dabei frei einstellbar, bei einem Abstand von
\num{0.2} ergibt sich die in Abbildung \ref{fig:orte} dargestellte Konfiguration.
Aus dem gegebenen Abstand $\delta$ wird zur Initialisierung die Anzahl an Ortsvektoren
mit der Funktion \texttt{sa\_number\_of\_pos\_vecs} bestimmt.
\begin{figure}
  \centering
  \includegraphics[height=8cm]{build/orte.pdf}
  \caption{Optimale Reihenfolge der Ortsvektoren.}
  \label{fig:orte}
\end{figure}
Die Reihenfolge der Ortsvektoren zufällig initialisiert.
Dies ergibt den in Abbildung \ref{fig:init} dargestellten Weg.
\begin{figure}
  \centering
  \includegraphics[height=8cm]{build/init.pdf}
  \caption{Initiale Reihenfolge der Ortsvektoren.}
  \label{fig:init}
\end{figure}

Für jeden vorgeschlagenen MC-Schritt muss die Weglänge berechnet werden.
Diese berechnet sich nach
\begin{equation*}
    L\!\left(\pi\right) =
    \sum_{i=2}^N \left|\vec{r}_{\pi\left(i\right)} - \vec{r}_{\pi\left(i-1\right)}\right|
    + \left|\vec{r}_{\pi\left(1\right)}-\vec{r}_{\pi\left(N\right)}\right|
\end{equation*}
in der Funktion \texttt{weglaenge}.

Schließlich wird in der Funktion \texttt{stimulated\_annealing} das Problem gelöst.
Dabei wird nach dem Metropolis-Algorithmus jeweils eine Vertauschung von zwei
Ortsvektoren in der Reihenfolge vorgeschlagen und angenommen,
wenn die neue Weglänge kleiner als die alte ist oder
$\exp\left(-\beta \: \increment L\right)$ kleiner als eine Zufallszahl
$x \in \left[0, 1\right]$ ist.
Hierbei ist besonders bei kleinen Abständen der Ortsvektoren $\delta$ die Anzahl an
Ortsvektoren groß und damit die Berechnung der Weglänge rechenintensiv.
Deshalb wird in jedem Schritt nur die neue Weglänge der vorgeschlagenen
Permutation berechnet.
Schließlich werden die Ortsvektoren und die beste gefundene Permutation in einer
Datei gespeichert.

\FloatBarrier
\subsection*{1.c)}

Bei der Lösung des Problems können vor allem zwei Parameter variiert werden.
Erstens ist dies die Anzahl an angebotenen Vertauschungen zweier Orte $S$ pro Temperatur
und zum anderen der Dämpfungsfaktor $d$,
mit welchem die Temperatur nach $S$ Vertauschungen multipliziert wird.
In Tabelle \ref{tab:Parametervariation} ist die optimale ermittelte Weglänge
für einen Durchlauf dargestellt.
Da der Seed nicht fest gesetzt ist, ergeben sich bei einem erneuten Ausführen des
Programms andere Werte.
Ebenso entsprechen die nachfolgenden Abbildungen der Endkonfigurationen
\ref{fig:d9S1e3} bis \ref{fig:d999S1e2}
nicht dem Durchlauf, bei welchem die Tabelle erstellt wurde.
\begin{table}
    \centering
    \caption{Optimale Weglängen bei Variation der Parameter.}
    \label{tab:Parametervariation}
    \begin{tabular}{
        c
        S[table-format=2.2]
        S[table-format=2.2]
        S[table-format=2.2]
        S[table-format=2.2]
        }
    \toprule
        {$d$} &
        \multicolumn{4}{c}{$S$} \\
        {} & {10} & {100} & {1000} & {10000} \\
    \midrule
        \num{0.9}   & 111.39 & 57.74 & 36.68 & 28.51 \\
        \num{0.99}  &  28.51 & 38.36 & 25.68 &       \\
        \num{0.999} &  25.68 & 30.38 &       &       \\
    \bottomrule
    \end{tabular}
\end{table}

Es zeigt sich, dass mit steigendem $d$ und $S$ die Weglänge immer weiter sinkt,
wie es auch zu erwarten war.
Aufgrund der zufällig gewählten Vertauschungen ist die gefundene Weglänge jedoch nicht
automatisch besser.
Um Rechenzeit zu sparen, wurden nicht alle Konfigurationen berechnet.

\begin{figure}
    \centering
    \includegraphics[height=8cm]{build/d9S1e3.pdf}
    \caption{Endkonfiguration bei $d = \num{0.9}$ und $S = \num{1000}$.}
    \label{fig:d9S1e3}
\end{figure}
\begin{figure}
    \centering
    \includegraphics[height=8cm]{build/d9S1e4.pdf}
    \caption{Endkonfiguration bei $d = \num{0.9}$ und $S = \num{10000}$.}
    \label{fig:d9S1e4}
\end{figure}
\begin{figure}
    \centering
    \includegraphics[height=8cm]{build/d99S1e3.pdf}
    \caption{Endkonfiguration bei $d = \num{0.99}$ und $S = \num{1000}$.}
    \label{fig:d99S1e3}
\end{figure}
\begin{figure}
    \centering
    \includegraphics[height=8cm]{build/d999S1e2.pdf}
    \caption{Endkonfiguration bei $d = \num{0.999}$ und $S = \num{100}$.}
    \label{fig:d999S1e2}
\end{figure}
