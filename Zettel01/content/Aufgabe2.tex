\section{Aufgabe 2}
\subsection{a)}
Das Problem zur Bestimmung der Ausgleichsgerade lässt sich als ein überbestimmtes
lineares Gleichungssystem der Form
\begin{equation}
  \tilde{A} \vec{x} = \vec{b}
  \label{eqn:Ausgangsproblem}
\end{equation}
mit
\begin{equation*}
  \tilde{A} =
  \begin{pmatrix}
    0 & 1 \\
    2,5 & 1 \\
    -6,3 & 1 \\
    4 & 1 \\
    -3,2 & 1 \\
    5,3 & 1 \\
    10,1 & 1 \\
    9,5 & 1 \\
    -5,4 & 1 \\
    12,7 & 1
  \end{pmatrix}
  \text{,} \ \vec{x} =
  \begin{pmatrix}
    m \\
    n \\
  \end{pmatrix}
  \ \text{und} \ \vec{b} =
  \begin{pmatrix}
    4 \\
    4,3 \\
    -3,9 \\
    6,5 \\
    0,7 \\
    8,6 \\
    13 \\
    9,9 \\
    -3,6 \\
    15,1 \\
  \end{pmatrix}
\end{equation*}
schreiben.

\subsection{b)}
Durch Multiplikation mi $\tilde{A}^T$ von links lässt sich \eqref{eqn:Ausgangsproblem}
in ein System mit einer quadratischen Matrix der Form
\begin{align}
  \tilde{P} \vec{x} &= \vec{b}' \\
  \label{eqn:symm-Problem}
  \tilde{P} &= \tilde{A}^T \tilde{A} \\
  \vec{b}' &= \tilde{A}^T \vec{b}
\end{align}
mit
\begin{equation*}
  \tilde{P} =
  \begin{pmatrix}
    482,98 & 29,2 \\
    29,2 & 10 \\
  \end{pmatrix}
  \ \text{und} \ \vec{b}' =
  \begin{pmatrix}
    541,22 \\
    54,6 \\
  \end{pmatrix}
\end{equation*}
überführen.

\subsection{c)}
Zur Lösung der Gleichung \eqref{eqn:symm-Problem} muss die Matrix $\tilde{P}$ invertiert
werden. Dazu wird die in Aufgabe 1 näher diskutierte LU-Zerlegung verwendet,
welche mit der \textsc{Eigen}-Bibliothek bestimmt wird und durch
\begin{equation*}
  \tilde{P} = \tilde{P_\text{LU}} \tilde{L} \tilde{U}
\end{equation*}
mit
\begin{equation*}
  \tilde{P_\text{LU}} =
  \begin{pmatrix}
    1 & 0 \\
    0 & 1 \\
  \end{pmatrix}
  \text{,} \ \tilde{L} =
  \begin{pmatrix}
    1 & 0 \\
    0,060458 & 1 \\
  \end{pmatrix}
  \ \text{und} \ \tilde{U} =
  \begin{pmatrix}
    482,98 & 29,2 \\
    0 & 8,23463 \\
  \end{pmatrix}
\end{equation*}
gegeben ist. Daraus lässt sich die Inverse mit
\begin{equation*}
  \tilde{P}^{-1} = \tilde{P_\text{LU}}^{-1} \tilde{L}^{-1} \tilde{U}^{-1} =
  \begin{pmatrix}
    0,00251436 & -0,00734192 \\
    -0,00734192 & 0,121438 \\
  \end{pmatrix}
\end{equation*}
berechnen. Damit ergeben sich der Schätzer für $\vec{x}$ zu
\begin{equation*}
  \vec{x} = \tilde{P}^{-1} \vec{b}' =
  \begin{pmatrix}
    0,959951 \\
    2,65694 \\
  \end{pmatrix}
  .
\end{equation*}

\subsection{d)}
Die Abbildung \ref{fig:leastsquares} wurde mit python erstellt.
\begin{figure}
  \centering
  \includegraphics[width=\textwidth]{build/leastsquares.pdf}  % [width=\textwidth]
  \caption{Datenpunkte mit Ausgleichsgerade nach least squares.}
  \label{fig:leastsquares}
\end{figure}
