\section*{Aufgabe 1}
% Das Bild wurde mit der vor-implementierten Methode eingelesen und es wurde
% eine SVD durchgeführt. Anschließend wurde eine Rang-$k$-Approximation durchgeführt.
% Die so entstandene Matrix wurde in einer txt-Datei gespeichert und mit einem
% \textsc{python}-Skript als Heatmap geplottet und in Abbildung \ref{fig:1}
% dargestellt.
% \begin{figure}
  % \centering
  % \begin{subfigure}{0.49\textwidth}
    % \includegraphics[width=\textwidth]{build/10.pdf}
    % \caption{$k = 10$}
    % \label{sub:10}
  % \end{subfigure}
  % \begin{subfigure}{0.49\textwidth}
    % \includegraphics[width=\textwidth]{build/20.pdf}
    % \caption{$k = 20$}
    % \label{sub:20}
  % \end{subfigure}
  % \\
  % \centering
  % \begin{subfigure}{0.49\textwidth}
    % \includegraphics[width=\textwidth]{build/50.pdf}
    % \caption{$k = 50$}
    % \label{sub:50}
  % \end{subfigure}
  % \caption{Heatmaps der Rang-$k$-Approximationen.}
  % \label{fig:1}
% \end{figure}
