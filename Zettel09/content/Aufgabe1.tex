\section*{Aufgabe 1}
\subsection*{a)}
Die DGL entspricht der eines harmonischen Oszillators mit dem zusätzlichen
Term $- \alpha \dot{x}$. Für $\alpha > 0$ entspricht das einem gedämpften
harmonischen Oszillator, so wie in Abbildung \ref{fig:gedämpft} dargestellt.

\begin{figure}
  \centering
  \includegraphics[scale=0.7]{build/aufg1_a1.pdf}
  \caption{Trajektorie für $\alpha = 0.1$.}
  \label{fig:gedämpft}
\end{figure}

Für $\alpha < 0$ entspricht die DGL einem Oszillator mit Motor, sodass eine
erzwungene Schwingung auftritt, wie in Abbildung \ref{fig:erzwungen}.

\begin{figure}
  \centering
  \includegraphics[scale=0.7]{build/aufg1_a2.pdf}
  \caption{Trajektorie für $\alpha = -0.1$.}
  \label{fig:erzwungen}
\end{figure}

Für $\alpha = 0$ schließlich sollte sich ein harmonischer Oszillator einstellen,
so wie in Abbildung \ref{fig:harmonisch} dargestellt.

\begin{figure}
  \centering
  \includegraphics[scale=0.7]{build/aufg1_a3.pdf}
  \caption{Trajektorie für $\alpha = 0$.}
  \label{fig:harmonisch}
\end{figure}

Alles in allem beschreibt $\alpha$ die Stärke der Dämpfung bzw. die Stärke des
Antriebs.

\subsection*{c)}
Bei einem oberen Zeitintervall von 200 und 3000 Schritten ist die Runge-Kutta-Methode
leicht schneller mit \SI{0.053875}{\s} gegenüber der AB-Methode mit \SI{0.055906}{\s}.

Wenn das Zeitintervall auf 2000 gestreckt und die Schrittweite auf 30000 erhöht wird,
dann ist die AB-Methode mit \SI{0.519792}{\s} schneller als die Runge-Kutta-Methode
mit \SI{0.543176}{\s}.

In der Vorlesung haben wir gelernt, dass die AB-Methode meist schneller ist als
die RK-Methode, vor allem, wenn die auszuwertende Funktion aufwendig zu berechnen
ist und keine Schrittweitenanpassung nötig ist. Diese Funktion ist nicht besonders
schwierig auszuwerten, deswegen wird die AB-Methode erst schneller, wenn man zu
langen Zeiten und vielen Schritten übergeht.
