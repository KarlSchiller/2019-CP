\section*{Aufgabe 2}
\subsection*{Aufgabenteil a)}
Das Gradienten- und das Konjugierte-Gradienten-Verfahren wurden implementiert.
Dabei wurde zur Bestimmung der optimalen Schrittweite das
Intervallhalbierungsverfahren, welches auch in der ersten Aufgabe implementiert
werden musste, verwendet. Die implementierten Funktionen wurden sodann an der
Rosenbrock-Funktion getestet. Der Contour-Plot der
Konjugierten-Gradienten-Methode befindet sich in Abbildung \ref{fig:1}. Dabei
entspricht die Colorbar den Funktionswerten an den jeweiligen Stellen. Dabei
musste die do-while-Schleifen angepasst werden, da ansonsten die neuen Schritte
gegen $\infty$ divergieren und damit kein Ergebnis erzielt werden kann. Wenn die
Bedingung in der do-while-Schleife \textsc{$((x_i.norm() < 1) || (i<12000))$}
gewählt wird, dann ergibt sich für
\begin{equation*}
  \vec{x}_{\symup{min}} = \begin{pmatrix}
    1.00234 \\
    0.133408
\end{pmatrix} \ \text{mit} \ f(\vec{x}_{\symup{min}}) = 75.90775847319088 \, .
\end{equation*}
Dies ist offensichtlich kein Minimum, leider ist uns der Fehler nicht bewusst,
der zum divergieren der Schritte führt. Der Contour-Plot wurde auch auf diese
Weise erstellt.

\begin{figure}
  \centering
  \includegraphics{build/conjugate.pdf}
  \caption{Contour-Plot des Konjugierten-Gradienten-Verfahrens, getestet an der
  Rosenbrock-Funktion.}
  \label{fig:1}
\end{figure}

 Die Konvergenz beim Gradienten-Verfahren ist wie erwartet sehr langsam, aus
diesem Grund wurde die Genauigkeit des Intervallhalbierungsverfahren auf
\num{e-6} statt wie bei der CG-Methode auf \num{e-8} gesetzt, um die Konvergenz
zu beschleunigen. Der Contour-Plot befindet sich in Abbildung
\ref{fig:gradient}. Da die Konvergenz sehr langsam ist, wurde nur jeder
\mbox{1000. Schritt} abgespeichert und nur jeder zehnte geplottet, da ansonsten
Speicherprobleme aufgetreten sind, wenn \textsc{np.meshgrid} aufgerufen wurde.
Dabei wurde als Abbruchkriterium $|\vec{g}_i| < 0.008$ gewählt, bei dem
CG-Verfahren ist es $|\vec{g}_{i+1}| \leq 0.009$. Dann konvergiert das
Gradienten-Verfahren gegen
\begin{equation*}
  \vec{x}_{\symup{min}} = \begin{pmatrix}
    0.991117 \\
    0.982278
\end{pmatrix} \ \text{mit} \ f(\vec{x}_{\symup{min}}) = 0.017687214165675095 \, .
\end{equation*}

\begin{figure}
  \centering
  \includegraphics{build/gradient.pdf}
  \caption{Contour-Plot des Gradienten-Verfahrens, getestet an der
  Rosenbrock-Funktion.}
  \label{fig:gradient}
\end{figure}

\subsection*{Aufgabenteil b)}
In diesem Aufgabenteil wurde mithilfe des CG-Verfahrens das Minimum der beschrieben Funktion
gesucht. Dabei fällt auf, dass das Verfahren nicht für den dritten Startwert funktioniert,
da die Einträge des ersten Gradientenvektors sich im Bereich von \num{e-29} bewegen,
also sehr klein und für den Rechner 0 sind. Damit kann dieses Verfahren für diesen
Startwert nicht angewendet werden. Für den ersten Startwert ergibt sich
\begin{equation*}
  \vec{x}_{\symup{min}} = \begin{pmatrix}
    1.68002 \\
    1.77503
\end{pmatrix} \ \text{mit} \ f(\vec{x}_{\symup{min}}) = 0.8569875533935767
\end{equation*}
und für den zweiten Startwert
\begin{equation*}
  \vec{x}_{\symup{min}} = \begin{pmatrix}
    -1.67661 \\
    -1.78004
\end{pmatrix} \ \text{mit} \ f(\vec{x}_{\symup{min}}) = 0.8570217685427779 \, .
\end{equation*}
Damit sind zwei Minima gefunden worden, möglicherweise existieren noch mehr.

\begin{figure}
  \centering
  \begin{subfigure}{0.49\textwidth}
    \centering
    \includegraphics[width=\textwidth]{build/b1.pdf}
    \caption{Erster Startwert.}
    \label{sub:1}
  \end{subfigure}
  \begin{subfigure}{0.49\textwidth}
    \centering
    \includegraphics[width=\textwidth]{build/b2.pdf}
    \caption{Zweiter Startwert.}
    \label{sub:2}
  \end{subfigure}
  \caption{Contourplots.}
  \label{fig:2}
\end{figure}
