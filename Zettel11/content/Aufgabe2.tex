\section*{Aufgabe 2}
Die Monte-Carlo-Simulation wurde implementiert. Zur Bestimmung der
Magnetisierung wurde
\begin{equation*}
  m = m_s \cdot \symup{tanh}(H)
\end{equation*}
genutzt, wobei $m_s$ die Sättigungsmagnetisierung, also maximale Magnetisierung
ist. Beim Durchlaufen des \texttt{C++}-Skriptes wird die Differenz zwischen
spin-up und spin-down in eine Datei geschrieben. diese wird dann normiert auf
den größten Wert, also die Sättigungsmagnetisierung. Damit ergibt sich die Kurve
in Abbildung \ref{fig:100000}
\begin{figure}
  \centering
  \includegraphics[scale=0.7]{build/aufg2.pdf}
  \caption{Magnetisierung bei 100000 Werten von $H$.}
  \label{fig:100000}
\end{figure}
Tatsächlich reichen auch schon 100 Werte von H, um zu einen vergleichbaren
Ergebnis zu kommen, wie Abbildung \ref{fig:100} zeigt.
\begin{figure}
  \centering
  \includegraphics[scale=0.7]{build/aufg2_100.pdf}
  \caption{Magnetisierung bei 100 Werten von $H$.}
  \label{fig:100}
\end{figure}
