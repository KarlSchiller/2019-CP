\section*{Aufgabe 1}
\subsection*{1.a)}
Das Runge-Kutta-Verfahren erster Ordnung wurde implementiert. Dann wurde das
Programm an einem harmonischen Oszillator getestet.
Für die Anfangsbedingungen
\begin{equation*}
  \vec{r}(0) = \begin{pmatrix}
  1 \\ 2 \\ 3 \\
\end{pmatrix} \ \text{und} \
\vec{v}(0) = \vec{0}
\end{equation*}
ergab sich die Trajektorie, die in Abbildung \ref{fig:harm} dargestellt ist. Es
ist zu sehen, dass eine harmonische Schwingung ausgeführt wird, da die jüngeren
Zeitschritte die älteren überlagern und dadurch sichtbar wird, dass eine
harmonische Schwingung auf dieser Trajektorie gegeben ist.

\begin{figure}
  \centering
  \includegraphics[scale=0.7]{build/a_harm.pdf}
  \caption{Trajektorie des harmonischen Oszillators bei $\vec{v} = \vec{0}$.}
  \label{fig:harm}
\end{figure}

Für die Anfangsbedingungen
\begin{equation*}
  \vec{r}(0) = \begin{pmatrix}
  1 \\ 0 \\ 3 \\
\end{pmatrix} \ \text{und} \
\vec{v}(0) = \begin{pmatrix}
  0 \\ 1 \\ 0 \\
\end{pmatrix}
\end{equation*}
ergibt sich die Trajektorie in Abbildung \ref{fig:unharm}. Es ist zu sehen, dass
sich eine Kreisbewegung einstellt, wie es für einen harmonischen Oszillator
zu erwarten ist, wenn Orts- und Geschwindigkeitsvektor senkrecht aufeinander stehen.
An dem Übergang zwischen blau und gelb ist zu sehen, wo die Bewegung beginnt
und dass sich der Kreis wieder schließt.

\begin{figure}
  \centering
  \includegraphics[scale=0.7]{build/a_unharm.pdf}
    \caption{Trajektorie des harmonischen Oszillators bei $\vec{v} \perp \vec{r}$.}
  \label{fig:unharm}
\end{figure}

\subsection*{1.b)}
Es stellte sich heraus, dass für 10 Schritte bzw. Schwingungen eine Schrittweite
von $h = \num{e-7}$ benötigt wurde, damit die angegebene Toleranzgrenze erfüllt
ist.

\subsection*{1.c)}
Bei 300 Schritten in einem Zeitintervall von $t = [0, 20]$ ergibt sich eine Abweichung
von der anfänglichen Gesamtenergie ab der sechsten Nachkommastelle.
