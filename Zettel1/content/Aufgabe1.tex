\section{Aufgabe 1}
\subsection{a)}
Die LU-Zerlegung ohne Pivotisierung wurde implementiert, das Ergebnis lautet
\begin{equation}
  L =
  \begin{pmatrix}
    1 & 0 & 0 & 0 \\
    -4 & 1 & 0 & 0 \\
    2 & 2 & 1 & 0 \\
    6 & 1 & 2 & 1 \\
  \end{pmatrix}
  \ \text{und} \ U =
  \begin{pmatrix}
    1 & 5 & -4 & 2 \\
    0 & 3 & -2 & 3 \\
    0 & 0 & 2 & -3 \\
    0 & 0 & 0 & 4 \\
  \end{pmatrix} \, .
\end{equation}
\subsection{b)}
Die LU-Zerlegung mit Pivotisierung wurde implementiert, allerdings sind die Ergebnisse
numerisch sehr instabil und starken Rundungsfehlern unterlegen,
deswegen wurden manche Ergebnisse vom Computer zu 0 gerundet,
allerdings erkennt man die Strukturen in den Matrizen
\begin{equation}
  L =
  \begin{pmatrix}
    1 & 0 & 0 & 0 \\
    -0,5 & 1 & 0 & 0 \\
    0 & 0.367347 & 1 & 0 \\
    0 & 0 & 0 & 1 \\
  \end{pmatrix}
  \ \text{und} \ U =
  \begin{pmatrix}
    -4 & 17 & 14 & 5 \\
    0 & 24,5 & -3 & 9,5 \\
    0 & 0 & -16,898 & 10,5102 \\
    0 & 0 & 0 & 0 \\
  \end{pmatrix} \, .
\end{equation}
Falls man die LU-Zerlegung mit Pivotisierung für diese Matrix mal mit \textsc{Eigen}
durchführt, zeigen sich Abweichungen bei manchen Matrixelementen. Die liegt daran,
dass wir mehr Rundungsfehler machen als die Methode der \textsc{Eigen}-Bibliothek.
Für die Permutationsmatrix ergibt sich
\begin{equation}
  P =
  \begin{pmatrix}
    0 & 0 & 0 & 1 \\
    0 & 0 & 1 & 0 \\
    1 & 0 & 0 & 0 \\
    0 & 1 & 0 & 0 \\
  \end{pmatrix} \, .
\end{equation}

\subsection{c)}
Mittels der Methode \textsc{Eigen::PartialPivLU::matrixLU} wurde die Matrix bestimmt,
die die Elemente der LU-Zerlegung enthält:
\begin{equation}
  M =
  \begin{pmatrix}
    4 & 2 & -4 \\
    -0,5 & 2 & -2 \\
    0,25 & 0,75 & 6,5 \\
  \end{pmatrix} \, .
\end{equation}
Dann wurde das Problem $PA = LU$ umgestellt zu
\begin{align}
  PA &= (A^{\symup T}P^{\symup T})^{\symup T} = LU \\
  \Leftrightarrow A^{\symup T}P^{\symup T} &= (LU)^{\symup T} \label{eqn:c} \, .
\end{align}
Dann wurde auf \eqref{eqn:c} die \textsc{solve}-Methode angewendet, um die Matrix
$P$ herauszufinden. Es ergibt sich
\begin{equation}
  P^{\symup T} = P =
  \begin{pmatrix}
    0 & 0 & 1 \\
    0 & 1 & 0 \\
    1 & 0 & 0 \\
  \end{pmatrix} \, .
\end{equation}
