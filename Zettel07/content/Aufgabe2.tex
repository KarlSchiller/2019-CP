\section*{Aufgabe 2}

\subsection*{a)}
Als Lösung bei $N$ Punkten wird ein $N$-Eck erwartet, dessen Seitenlänge alle
gleich sind, also
\begin{equation*}
  \vec{d}_\text{i} = \vec{d} \quad \forall i=\left\{1, ..., N\right\}
\end{equation*}
gilt (z.B. ein Quadrat für $N = 4$).
Bei $N \rightarrow \infty$ ergibt sich als optimaler Polygonzug ein Kreis.
\subsection*{b)}

In Abbildung \ref{fig:1} ist ein Plot der Startkonfiguration zu sehen. Es wurde
versucht, die Augmented-Langrangian-Method mit dem BFGS Algorithmus zu
programmieren. Leider oszilliert dabei die Abbruchbedingung der
do-while-Schleife der minimize-Funktion, die im Zuge von BFGS aufgerufen wird,
sodass eine Endlosschleife entsteht. Aus diesem Grund wurde der Aufruf der
$\textsc{opti\_poly}$-Methode auskommentiert, da ansonsten das Programm nicht
durchläuft. Falls es durchlaufen würde, sollte der Flächeninhalt $\pi$ betragen.

\begin{figure}
  \centering
  \includegraphics[scale=0.7]{build/aufg2-l1-start.pdf}
  \caption{Plot für die Startkonfiguration mit $l = 1$.}
  \label{fig:1}
\end{figure}

\subsection*{c)}
Für Aufgabenteil c) wurde der Plot mit der Startkonfiguration erstellt. Er ist
in Abbildung \ref{fig:2} zu sehen.

\begin{figure}
  \centering
  \includegraphics[scale=0.7]{build/aufg2-l2-start.pdf}
  \caption{Plot für die Startkonfiguration mit $l = 2$.}
  \label{fig:2}
\end{figure}
