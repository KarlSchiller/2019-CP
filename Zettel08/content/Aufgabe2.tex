\section*{Aufgabe 2}

\subsection*{2.a)}
Das Runge-Kutta-Verfahren aus Aufgabe 1 wurde dazu verwendet, um das Keppler-Problem
zu lösen.
Für die Startwerte
\begin{equation*}
    \vec{r}\!\left(0\right) =
    \begin{pmatrix}
        1 \\
        0 \\
        0 \\
    \end{pmatrix}
    \quad
    \vec{v}\!\left(0\right) =
    \begin{pmatrix}
        0 \\
        0.5 \\
        0.8 \\
    \end{pmatrix}
\end{equation*}
ergibt sich die in Abbildung \ref{fig:2_ellipse} dargestellte Ellipse.
\begin{figure}
  \centering
  \includegraphics[scale=0.7]{build/aufg2_a_ellipse.jpg}
  \caption{Eine Keppler-Ellipse.}
  \label{fig:2_ellipse}
\end{figure}

Bei kleinen Anfangsgeschwindigkeiten führt das Teilchen keine Ellipsenbahn mehr aus,
sondern schwingt lediglich wie ein harmonischer Oszillator (vergleiche Aufgabe 1.a)
wie in Abbildung \ref{fig:2_schmetterling} dargestellt.
\begin{figure}
  \centering
  \includegraphics[scale=0.7]{build/aufg2_a_schmetterling.jpg}
  \caption{Eine Keppler-Ellipse für sehr kleine Anfangsgeschwindigkeiten.}
  \label{fig:2_schmetterling}
\end{figure}

\subsection*{2.b)}
Für jeden Zeitschritt wurde die Energie und der Drehimpulsvektor berechnet.
In den Abbildungen \ref{fig:2_energie} und \ref{fig:2_drehimpuls} sind die Abweichungen
der beiden Größen von den Startwerden dargestellt.
\begin{figure}
  \centering
  \includegraphics[scale=0.7]{build/aufg2_b_energie.jpg}
  \caption{Die Abweichung der Energie vom Startwert.}
  \label{fig:2_energie}
\end{figure}
\begin{figure}
  \centering
  \includegraphics[scale=0.7]{build/aufg2_b_drehimpuls.jpg}
  \caption{Die Abweichung des Drehimpulsvektors vom Startwert.}
  \label{fig:2_drehimpuls}
\end{figure}
Es zeigt sich, dass Energieabweichung mit der Zeit zunimmt. Dies ist eine Motivation für
andere, energieerhaltende Verfahren.
Der Drehimpulsvektor weicht mit der Zeit in dem selben Maße ab, wie es auch die Energie
tut. Lediglich für die Komponente, bei welcher der Drehimpuls Null beträgt, zeigen
sich deutlich kleinere Abweichungen am numerischen Limit.

\subsection*{2.c)}
Nun wird das Potential zu
\begin{equation*}
    V\!\left(\vec{r}\right) =
    - \frac{m G}{\left|\vec{r}\right|^\alpha}
\end{equation*}
verändert.
Das Kraftfeld ist durch
\begin{equation*}
    \vec{F}\!\left(\vec{r}\right)
    = - \vec{\nabla} V\!\left(\vec{r}\right)
\end{equation*}
gegeben. Damit ergibt sich beispielhaft für die $x$-Komponente
\begin{align*}
    F_\text{x}\!\left(\vec{r}\right)
    &= -\frac{\partial}{\partial x}
        \left[- \frac{m G}{\left(\sqrt{x^2 + y^2 + z^2}\right)^\alpha}\right] \\
    &= m G \frac{\partial}{\partial x}
        \left[\left(x^2 + y^2 + z^2\right)^{-\frac{\alpha}{2}}\right] \\
    &= m G \left(-\frac{\alpha}{2}\right)
        \left(x^2 + y^2 + z^2\right)^{-\frac{\alpha}{2}-\frac{2}{2}}
        2 x \\
    &= - m G \alpha
        \frac{x}{\left(\sqrt{x^2 + y^2 + z^2}\right)^{\alpha+2}} \\
\end{align*}
und somit
\begin{equation*}
    \vec{F}\!\left(\vec{r}\right)
    = -\alpha m G \frac{\vec{r}}{\left|\vec{r}\right|^{\alpha +2}}
\end{equation*}

In den Abbildungen \ref{fig:2_alpha11} und \ref{fig:2_alpha09} sind die sich ergebenden
Trajektorien für $\alpha = \num{1.1}$ und \num{0.9} dargestellt.
Eine Verformung der Ellipse bzw. keine geschlossene Bahn ist erkennbar.
\begin{figure}
    \centering
    \includegraphics[scale=0.7]{build/aufg2_c_11.jpg}
    \caption{Trajektorie bei $\alpha = \num{1.1}$.}
    \label{fig:2_alpha11}
\end{figure}
\begin{figure}
    \centering
    \includegraphics[scale=0.7]{build/aufg2_c_09.jpg}
    \caption{Trajektorie bei $\alpha = \num{0.9}$.}
    \label{fig:2_alpha09}
\end{figure}
