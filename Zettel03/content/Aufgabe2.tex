\section*{Aufgabe 2}

\subsection*{2.a)}

Die Schrödingergleichung des anharmonischen Oszillators
\begin{equation*}
  \left[-\frac{\hbar^2}{2 m}\partial^2_\text{x} +
  \frac{1}{2} m \omega^2 x^2 + \lambda x^4\right] \psi\!\left(x\right)
  = E \psi\!\left(x\right)
\end{equation*}
wird durch eine Koordinatentransformation
\begin{equation*}
  x = \alpha \xi\text{,} \quad E = \beta \varepsilon\text{,} \quad \lambda = \gamma \tilde{\lambda}
\end{equation*}
in eine dimensionslose Form gebracht. Einsetzen ergibt
\begin{align*}
  \left[-\frac{\hbar^2}{2 m} \partial^2_\xi + \frac{1}{2} m \omega^2 \alpha^2 \xi^2
  + \gamma \tilde{\lambda} \alpha^4 \xi^4\right] \psi\!\left(\xi\right)
  &= \beta \varepsilon \psi\!\left(\xi\right) \\
  \Leftrightarrow \left[-\partial^2_\xi +
  \underbracket{\frac{m^2 \omega^2}{\hbar^2} \alpha^2}_{\Rightarrow\left|\alpha\right|=\frac{\hbar}{\m \omega}} \xi^2
  + \frac{2 m}{\hbar^2} \gamma \tilde{\lambda} \alpha^4 \xi^4\right] \psi\!\left(\xi\right)
  &= \underbracket{\frac{2 m}{\hbar^2} \beta}_{\Rightarrow \beta=\frac{\hbar^2}{2 m}}
  \varepsilon \psi\!\left(\xi\right) \\
  \Leftrightarrow \left[-\partial^2_\xi + \xi^2
  + \underbracket{\gamma \frac{2 \hbar^4 m}{\hbar^2 m^4 \omega^4}}_{\Rightarrow \gamma = \frac{m^3 \omega^4}{2 \hbar^2}}
  \tilde{\lambda} \xi^4\right] \psi\!\left(\xi\right)
  &= \varepsilon \psi\!\left(\xi\right) \\
  \Leftrightarrow \left[-\partial^2_\xi + \xi^2 + \tilde{\lambda} \xi^4\right] \psi\!\left(\xi\right)
  &= \varepsilon \psi\!\left(\xi\right)
\end{align*}

\subsection*{2.b)}

Der Ortsraum wird mit auf einem Interval $\xi = \left[-L; L\right]$ diskretisiert
durch $\xi_\text{n} = n \increment\xi$.
Somit ergibt sich der Hamiltonoperator in Ortsdarstellung zu
\begin{equation*}
  \Braket{\xi_\text{n} | \hat{H} | \xi_\text{m}} =
  -\frac{1}{\left(\increment \xi\right)}
  \left(\delta_\text{n,m-1} + \delta_\text{n,m+1} -2 \delta_\text{n,m} \right)
  + \left[\left(\increment \xi\right)^2 n^2 +
  \tilde{\lambda} \left(\increment \xi\right)^4 n^4\right] \delta_\text{n,m}
\end{equation*}

Für $\tilde{\lambda} = \num{0.2}$, $L = \num{10}$ und
$\left(\increment \xi\right) = \num{0.1}$ egeben sich die 10 niedrigsten Eigenwerte
durch Diagonalisierung zu
\begin{gather*}
  \num{1.1174} \quad \num{3.53373} \quad \num{6.26136} \quad \num{9.22317} \quad \num{12.3776} \\
  \num{15.6969} \quad \num{19.1609} \quad \num{22.7538} \quad  \num{26.463} \quad \num{30.2779}
\end{gather*}

\subsection*{2.c)}

Nun werden die Eigenwerte noch einmal anders berechnet.
Dazu wird ein Unterraum der Besetzungszahl-Eigenzustände des ungestörten Problems betrachtet.

Die Matrixelemente des ungestörten Hamiltons $\hat{H}_0$ sind bekannt als
\begin{equation*}
  \Braket{n | \hat{H}_0 | m} = \left(2n+1\right) \delta_\text{n,m}.
\end{equation*}
Der Störterm $\hat{H}_\text{st}$ ergibt sich laut Aufgabenstellung zu
\begin{align}
  4 \tilde{\lambda} \Braket{n | \xi^4 | m}
  &= \sqrt{m \left(m-1\right) \left(m-2\right) \left(m-3\right)}\:\delta_\text{n,m-4}\nonumber\\
  &+ \sqrt{\left(m+1\right) \left(m+2\right) \left(m+3\right) \left(m+4\right)}
    \:\delta_\text{n,m+4}\nonumber\\
  &+ \sqrt{m \left(m-1\right)}\left(4m-2\right)\delta_\text{n,m-2}\nonumber\\
  &+ \sqrt{\left(m+1\right) \left(m+2\right)}\left(4m+6\right)\delta_\text{n,m+2}\nonumber\\
  &+ \left(6m\left(m+1\right)+3\right)\delta_\text{n,m}
  .
  \label{eqn:Stoerterm}
\end{align}
Damit ist der gesamte Hamilton durch $\hat{H} = \hat{H}_0 + \hat{H}_\text{St}$ gegeben.

Zunächst wird gezeigt, dass der $\xi^4$-Term hermitesch ist, dass also
\begin{equation*}
  \Braket{n | \hat{H}_\text{St} | m} = \Braket{m | \hat{H}_\text{St} | n}
\end{equation*}
gilt. Die Einträge in $\hat{H}_\text{St}$ sind reell,
daher entfällt die komplexe Konjugation.
\begin{align*}
  4 \Braket{n | \xi^4 | m}
  &= \sqrt{n \left(n-1\right) \left(n-2\right) \left(n-3\right)}\:\delta_\text{m,n-4}\\
  &+ \sqrt{\left(n+1\right) \left(n+2\right) \left(n+3\right) \left(n+4\right)}
    \:\delta_\text{n,m+4} \\
  &+ \sqrt{n \left(n-1\right)}\left(4n-2\right)\delta_\text{m,n-2} \\
  &+ \sqrt{\left(n+1\right) \left(n+2\right)}\left(4n+6\right)\delta_\text{m,n+2} \\
  &+ \left(6n\left(n+1\right)+3\right)\delta_\text{n,m}
\end{align*}
Nun wird für den Index $n$ der Index $m$ eingesetzt. Dabei ist die Relation zwischen
den Indices durch das jeweilige Koneckerdelta gegeben.
\begin{align*}
  4 \Braket{n | \xi^4 | m}
  &= \sqrt{\left(m+4\right) \left(m+4-1\right) \left(m+4-2\right) \left(m+4-3\right)}\:\delta_\text{m,n-4}\\
  &+ \sqrt{\left(m-4+1\right) \left(m-4+2\right) \left(m-4+3\right) \left(m-4+4\right)}
    \:\delta_\text{n,m+4} \\
  &+ \sqrt{\left(m+2\right) \left(m+2-1\right)}\left(4\left(m+2\right)-2\right)\delta_\text{m,n-2} \\
  &+ \sqrt{\left(m-2+1\right) \left(m-2+2\right)}\left(4\left(m-2\right)+6\right)\delta_\text{m,n+2} \\
  &+ \left(6m\left(m+1\right)+3\right)\delta_\text{n,m} \\
  \\
  &= \sqrt{\left(m+1\right) \left(m+2\right) \left(m+3\right) \left(m+4\right)}\:\delta_\text{m,n-4}\\
  &+ \sqrt{m \left(m-1\right) \left(m-2\right) \left(m-3\right)}
    \:\delta_\text{n,m+4} \\
  &+ \sqrt{\left(m+1\right) \left(m+2\right)}\left(4m+6\right)\delta_\text{m,n-2} \\
  &+ \sqrt{m \left(m-1\right)}\left(4m-2\right)\delta_\text{m,n+2} \\
  &+ \left(6m\left(m+1\right)+3\right)\delta_\text{n,m} \\
  \\
  &= 4 \Braket{m | \xi^4 | n}
\end{align*}
Das letzte Gleichheitszeichen gilt aufgrund
\begin{align*}
  \delta_\text{n,m-4} &=
  \begin{cases}
    1 & n=m-4 \\
    0 & \text{sonst}
  \end{cases}
  =
  \begin{cases}
    1 & m=n+4 \\
    0 & \text{sonst}
  \end{cases}
  = \delta_\text{m,n+4} \\
  &\text{und analog} \\
  \delta_\text{n,m+4} &= \delta_\text{m,n-4} \\
  \delta_\text{n,m-2} &= \delta_\text{m,n+2} \\
  \delta_\text{n,m+2} &= \delta_\text{m,n-2}
  .
\end{align*}

Für $\tilde{\lambda} = \num{0.2}$ und $N = \num{50}$ bei
$n \in \left\{0;1;...;N\right\}$ ergeben sich die 10 niedrigsten Eigenwerte zu
\begin{gather*}
  \num{1.11829} \quad \num{3.53901} \quad \num{6.27725} \quad \num{9.25777} \quad \num{12.4406} \\
  \num{15.7995} \quad \num{19.3157} \quad \num{22.9746} \quad \num{26.7649} \quad \num{30.6773}
\end{gather*}

\subsection*{2.d)}

Es wird systematisch $N$ zwischen \num{10} und \num{200} in \num{10}er-Schritten
variiert. Somit verändert sich die Diskretisierung mit
\begin{equation*}
  \increment \xi = \frac{2 L}{N}.
\end{equation*}
In den Abbildungen \eqref{eqn:energy-vergleich} und \eqref{eqn:position-vergleich}
sind die Abweichungen der berechneten Eigenwerte vom berechneten Eigenwert bei
$N = \num{200}$ dargestellt.

Die Berechnungen im Unterraum der ersten $N$ Energieeigenzustände ist deutlich genauer,
was sich daran erkennen lässt, dass die Abweichung zum genauesten berechneten
Eigenwert bei deutlich kleineren $N$ kleiner sind bis zu $10^{-13}$, während
die Berechnung mit der Diskretisierung des Ortsraumes nur auf $10^{-4}$ kommt.
Dies könnte daran liegen, bei der Energiedarstellung vor allem höhere Zustände für
die Berechnung der Grundzustandsenergie weniger ins Gewicht fallen, sodass hier
eine schnelle Konvergenz erreicht wird.
Im Ortsraum wurde die zweite Ableitung numerisch genähert und diese könnte deutlich
langsamer konvergieren.

\begin{figure}
  \centering
  % \includegraphics[width=\texwidth]{build/aufg2-energy.pdf}
  \includegraphics{build/aufg2-energy.pdf}
  \caption{Differenz der berechneten Eigenwerte und der Eigenwerte bei $N = \num{200}$.
  Hierbei steht EW 0 für den niedrigsten Eigenwert.}
  \label{eqn:energy-vergleich}
\end{figure}
\begin{figure}
  \centering
  % \includegraphics[width=\texwidth]{build/aufg2-position.pdf}
  \includegraphics{build/aufg2-position.pdf}
  \caption{Differenz der berechneten Eigenwerte und der Eigenwerte bei $N = \num{200}$.
  Hierbei steht EW 0 für den niedrigsten Eigenwert.}
  \label{eqn:position-vergleich}
\end{figure}
