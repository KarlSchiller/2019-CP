\section*{Aufgabe 1}
Das Intervallhalbierungs-/Bisektionsverfahren und das Newtonverfahren werden
mit der Funktion
\begin{equation*}
  f\!\left(x\right) = x^2 - 2
\end{equation*}
und einer Genauigkeit von $\varepsilon = 10^9$ getestet.

Das Intervallhalbierungsverfahren mit den Startpunkten
\begin{align*}
  a &= \num{-0.5} \\
  b &= \num{-0.1} \\
  c &= \num{2} \\
\end{align*}
ergibt den in Abbildung \ref{fig:A1-Bisektionsverfahren} dargestellten Verlauf.
In jedem Schritt wird das Intervall halbiert, was sich in dem linearem Verlauf
der Intervalllänge wiederspiegelt.
\begin{figure}
  \centering
  \includegraphics[width=.8\textwidth]{build/aufg1-bisection.pdf}
  \caption{Bisektionsverfahren.}
  \label{fig:A1-Bisektionsverfahren}
\end{figure}
Das in Abbildung \ref{fig:A1-Newtonverfahren} dargestellte Newtonverfahren approximiert
die Funktion durch eine quadratische Funktion,
sodass die hier verwendete Funktion sehr schnell minimiert wird.
Aufgrund der Ungenauigkeit bei der Bestimmung der ersten und vor allem zweiten
Ableitung konvergiert das Verfahren nicht bis zur angegebenen Genauigkeitsgrenze.
\begin{figure}
  \centering
  \includegraphics[width=.8\textwidth]{build/aufg1-newton.pdf}
  \caption{Newtonverfahren.}
  \label{fig:A1-Newtonverfahren}
\end{figure}
