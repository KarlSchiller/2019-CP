\section*{Aufgabe 1}
\subsection*{Aufgabenteil a) + b)}
Um das zu berechnende Integral einheitenlos zu machen, werden folgende Transformationen
eingeführt
\begin{align*}
  x^+ &\mapsto \frac{x}{a} \\
  \Phi^{'} &\mapsto \Phi  \frac{4\, \pi \, \epsilon_0}{\rho_0 \, a^2} \, .
\end{align*}
Dann wurde die dreidimensionale Integration implementiert. In Abbildung \ref{subfig:a}
ist der Plot für das elektrostatische Potential außerhalb des Würfels zu sehen.
In Abbildung \ref{subfig:b} ist das Potential innerhalb des Würfels abgebildet.
\begin{figure}
  \centering
  \begin{subfigure}{0.49\textwidth}
    \centering
    \includegraphics[width=\textwidth]{build/aufg_a.pdf}
    \caption{Plot zum Aufgabenteil a).}
    \label{subfig:a}
  \end{subfigure}
  \begin{subfigure}{0.49\textwidth}
    \centering
    \includegraphics[width=\textwidth]{build/aufg_b.pdf}
    \caption{Plot zum Aufgabenteil b).}
    \label{subfig:b}
  \end{subfigure}
  \caption{Plots für die erste Ladungsverteilung innerhalb und außerhalb
  des Würfels.}
  \label{fig:1}
\end{figure}

\subsection*{Aufgabenteil c)}
In den Abbildungen \ref{subfig:c_1} und \ref{subfig:c_2} sind die Plots zu
Aufgabenteil c) zu sehen.
\begin{figure}
  \centering
  \begin{subfigure}{0.49\textwidth}
    \centering
    \includegraphics[width=\textwidth]{build/aufg_c_1.pdf}
    \caption{Plot zum Aufgabenteil c) außerhalb des Würfels.}
    \label{subfig:c_1}
  \end{subfigure}
  \begin{subfigure}{0.49\textwidth}
    \centering
    \includegraphics[width=\textwidth]{build/aufg_c_2.pdf}
    \caption{Plot zum Aufgabenteil c) innerhalb des Würfels.}
    \label{subfig:c_2}
  \end{subfigure}
  \caption{Plots für die zweite Ladungsverteilung innerhalb und außerhalb
  des Würfels.}
  \label{fig:2}
\end{figure}
