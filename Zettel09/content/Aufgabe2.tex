\section*{Aufgabe 2}
Der Koordinatenursprung liegt im Kern der Erde, somit ergibt sich der Startvektor für
r zu
\begin{equation*}
  \begin{pmatrix}
    0 \\
    R+h_0
  \end{pmatrix} \, .
\end{equation*}

Es wurde eine Runge-Kutta-Verfahren 4. Ordnung implementiert, um die
Bewegungsgleichung der Raumkapsel zu lösen. Dabei wurde das Verfahren solange
angewendet, bis der errechnete Ortsvektor $\leq$ dem Erdradius war.
Währenddessen wurde nach Vorschrift in der Vorlesung die Schrittweite angepasst.
Da die Alternativ-Methode zur \textsc{pow}-Methode leider nicht funktioniert
hat, dauert das Programm etwas länger zum Durchlaufen. Dieses angepasste
Runge-Kutta wurde für die verschiedenen $\vec{v}_0$ mit einer Methode
aufgerufen, die den $2d$-dimensionalen Ergebnisvektor $y$ wieder in die
Einzelgrößen $r$ und $v$ zerlegt und die Größe

\begin{equation*}
  \frac{<\vec{r}, \vec{v}>}{|\vec{r}|}
\end{equation*}

bestimmt. Da nach dem Betrag der vertikalen Geschwindigkeit gefragt ist, wird
das Skalarprodukt zwischen dem Geschwindigkeitsvektor und dem Normalenvektor gebildet,
der in unserem Modell der Erde als perfekte Kugel dem Ortsvektor entspricht. Zusätzlich
wird das ganze mit dem Betrag des Normalenvektors noch normiert. Leider kommen bei
diesem Vorgehen die in Abbildung \ref{fig:norm} dargestellten Werte heraus.
Mir erscheint es schwierig, darauf jetzt ein Minimierungsverfahren anzuwenden,
welches den minimalen Startwert $v_0$ bestimmt. Ich vermute, dass in den bisherigen Schritten
etwas falsch gelaufen ist, finde aber leider den Fehler nicht.

\begin{figure}
  \centering
  \includegraphics[scale=0.7]{build/aufg2_a.pdf}
  \caption{Betrag der vertikalen Geschwindigkeit.}
  \label{fig:norm}
\end{figure}

Prinzipiell hätte ich wohl das Bisektionsverfahren verwendet, um den passenden
Anfangswert herauszufinden. In der a) für die positive, und in der b) für die negative
Funktion. Bei der b) wäre wahrscheinlich $v_0 = 0$ nach meiner Schätzung herausgekommen,
da auf diese Weise der Weg zum Erdboden minimiert wird und somit die Luftreibung
weniger den Fall bremsen kann.

Der Grund, warum Schrittweitenanpassung sinnvoll ist, ist der, dass die Luftreibung
höhenabhängig ist und auch die Geschwindigkeit stark variiert und sich somit die
DGL stark verändert, also an bestimmen Stellen nicht besonders glatt ist. Aus
diesem Grund ist es sinnvoll, den Vorteil von Runge-Kutta auszunutzen, dass $h$
zwischen einzelnen Schritten variiert werden kann, um flache Bereiche der
Funktion schnell (sprich: mit großer Schrittweite) zu durchlaufen und um schnell
variierende Bereiche langsam (sprich: mit kleiner Schrittweite) zu durchlaufen.
