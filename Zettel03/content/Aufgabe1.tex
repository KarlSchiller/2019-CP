\section*{Aufgabe 1}
\subsection*{Aufgabenteil a)}
Als Ansatz wurde Lagrange gewählt mit der folgenden Lagrange-Funktion
\begin{equation*}
  L = \sum_{i=1}^N \left(\frac{1}{2} m_i \dot{u_i}^2 - \frac{1}{2} k_i (u_{i+1}-u_i)^2 \right) \, ;
\end{equation*}
dabei sind $m_i$ die Massen, $k_i$ die Federkonstanten und $u_i$ die Positionen der Massen
in Normalkoordinaten.
Mit der Euler-Langrange-Gleichung
\begin{equation*}
  0 = \frac{\symup d}{\symup dt} \frac{\partial L}{\partial \dot{u_i}} - \frac{\partial L}{\partial u_i}
\end{equation*}
folgt
\begin{align*}
  \frac{\symup d}{\symup dt} m_i \dot{u_i} &= -k_{i-1}(u_i - u_{i-1}) + k_i(u_{i+1}-u_i) \\
  \Leftrightarrow \ m_i \ddot{u}_i &= k_{i-1} u_{i-1} - k_{i-1} u_i + k_i u_{i+1} - k_i u_i
\end{align*}
