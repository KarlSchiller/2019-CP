\section*{Aufgabe 2}

In der Datei \texttt{aufgabe2.cpp} befindet sich ein Versuch der Implementierung der FFT
unter Verwendung des ersten Algorithmus ohne Speicheroptimierung aus der Vorlesung.
Ebenfalls implementiert ist eine Methode \texttt{fft}, welche die anschließende
Verschiebung wie in Aufgabenteil b) beschrieben darstellt.

Leider befindet sich ein/mehrere Fehler in der Datei, sodass diese Methoden keine
sinnvollen Ergebnisse liefern. Auch nach längerer Fehleranalyse ließ sich dies nicht
ändern.

Der Vollständigkeit halber sei hier noch die Abbildung \ref{fig:test} aufgeführt,
welche sich mit der unnormierten Gaußfunktion in Aufgabenteil b) als Test ergibt.

\begin{figure}
  \centering
  \includegraphics[width=.8\textwidth]{build/test.pdf}
  \caption{Theoretisch sollte dies die Fouriertransformierte der Gaußfunktion
  sein, also wieder eine Gaußfunktion mit Standardabweichung \num{1}.}
  \label{fig:test}
\end{figure}
